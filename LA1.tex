\documentclass[paper=a4,fontsize=11pt]{scrartcl} % KOMA-article class

\documentclass{article}
\usepackage{graphicx}
\usepackage{caption}
\usepackage[utf8]{inputenc}
\usepackage[textwidth=17cm,textheight=25cm]{geometry}
\usepackage{amsmath}


\title{An example of using linear algebra}

\begin{document}

\maketitle

The presenting question is simple. Given an arbitrary charge distribution in vacuum, how do we calculate the resulting electric field $E$?

Define a vector $\mathbf{E}$ the electric field, a scalar $\phi$ the electric potential, and a scalar $\rho$ the charge distribution, i.e. where the charge is. By definition, the electric field is the negative gradient of the electric potential,
\begin{equation}\label{potential}
-\nabla \phi = \mathbf{E}.
\end{equation}
You must have seen a similar expression between the gravitational force and the gravitational potential from your high school. They are mathematically the same. 

A physical law, called Maxwell's equations, says that in vacuum,
\begin{align} \label{maxwell}
    \nabla\cdot\mathbf{E} &= \frac{\rho}{\epsilon_0},
\end{align}
where $\epsilon_0$ is just a physical constant that we do not need care too much. Substituting $eq.[\ref{potential}]$ into $eq.[\ref{maxwell}]$, we have
\begin{align}\label{maxwell2}
    -\nabla^2 \phi &= \frac{\rho}{\epsilon_0}.
\end{align}

So far, I haven't tell you what $\nabla$ represents, but in 1-D space it is equivalent to the operation of computing the derivative, i.e. $\nabla\equiv\frac{d}{dx}$, so $eq.[\ref{maxwell}]$ becomes a relatively more friendly form,
\begin{align} \label{maxwell3}
-\frac{d^2\phi}{dx^2} &= \frac{\rho}{\epsilon_0}.
\end{align}
Now we have an ordinary differential equation, and what meaning does it have? It means that if you give me a function $\rho(x)$ that describes where charges are located in a 1-D space, I can solve the differential equation $eq.[\ref{maxwell3}]$ for $\phi(x)$, and tell you the electric field $\mathbf{E}(x)$ at any point in the 1-D space!

How do we solve this equation? Easy, you just need to integrate the equation twice to get an expression for $\phi$! However, what if $\rho(x)$ is very nasty that you cannot integrate it by hand? One tricky way is to solve it NUMERICALLY, which almost always needs the aid of a computer. I will show you how.

Imagine a straight line of finite length $L$, and we can cut the line into pieces of equal lengths, say $h$. we now have $n=\frac{L}{h}$ pieces of line, and $N=n+1$ nodes. Denote the position of nodes by $x_0$, $x_1$, ..., $x_N$, and the values of them are $0$, $h$, ..., $Nh$. We can denote the electric potential $\phi(x)$ at the nodes as $\phi_0$, $\phi_1$, ..., $\phi_N$, which are what we want to find. If you give me a charge distribution function $\rho(x)$, I can certainly calculate the charge at the nodes: $\rho(x_0)$, $\rho(x_1)$, ..., $\rho(x_N)$. Let us write them down as vectors for now, and we will use them later.

\begin{equation}
\mathbf{x}=
    \begin{pmatrix}
    x_0 \\
    x_1 \\
    x_2 \\
    ... \\
    x_N
    \end{pmatrix}=
    \begin{pmatrix}
    0 \\
    h \\
    2h \\
    ... \\
    Nh
    \end{pmatrix},
    \mathbf{\Phi}=
    \begin{pmatrix}
    \phi(x_0) \\
    \phi(x_1) \\
    \phi(x_2) \\
    ... \\
    \phi(x_N)
    \end{pmatrix}=
    \begin{pmatrix}
    \phi_0 \\
    \phi_1 \\
    \phi_2 \\
    ... \\
    \phi_N
    \end{pmatrix},
    \mathbf{b}=
    \begin{pmatrix}
    \rho(x_0) \\
    \rho(x_1) \\
    \rho(x_2) \\
    ... \\
    \rho(x_N)
    \end{pmatrix}=
    \begin{pmatrix}
    \rho_0 \\
    \rho_1 \\
    \rho_2 \\
    ... \\
    \rho_N
    \end{pmatrix}
\end{equation}


Recall the definition of a derivative:
\begin{equation}
\frac{dy}{dx} = \lim\limits_{h \to 0}\frac{y(x+\frac{h}{2})-y(x-\frac{h}{2})}{h},
\end{equation}
and the second derivative:
\begin{align}
    \frac{d^2y}{dx^2} &= \lim\limits_{h \to 0}\frac{[y(x+h) - y(x)] - [y(x)-y(x-h)]}{h^2} \\
    &=  \lim\limits_{h \to 0}\frac{y(x+h) - 2y(x) + y(x-h)}{h^2}.
\end{align}
If $h$ is small enough, we can approximate the second derivative with 
\begin{align}
    \frac{d^2y}{dx^2} \approx  \frac{y(x+h) - 2y(x) + y(x-h)}{h^2}.
\end{align}
We can obtain the discretised form of the second derivative:
\begin{align}\label{difference_matrix}
    \frac{d^2\phi_i}{dx^2} &\approx  \frac{\phi(x_i+h) - 2\phi(x_i) + \phi(x_i-h)}{h^2} \\
    &\approx \frac{\phi_{i+1} - 2\phi_i + \phi_{i-1}}{h^2}.
\end{align}

The original differential equation $eq.[\ref{maxwell3}]$ can be written down in the matrix form:
\begin{align}\label{matrix_equation}
    \mathbf{A}\mathbf{\Phi} = \mathbf{b},
\end{align}
where $\mathbf{A}$ is defined as an (N+1) by (N+1) matrix:
\begin{equation}
\mathbf{A}=\frac{\epsilon_0}{h^2}
    \begin{pmatrix}
    -2 & 1 & 0 & 0 & ... & 0 & 0 & 0 & 0 \\
    1 & -2 & 1 & 0 & ... & 0 & 0 & 0 & 0 \\
    0 & 1 & -2 & 1 & ... & 0 & 0 & 0 & 0 \\
    0 & 0 & 1 & -2 &  ... & 0 & 0 & 0 & 0\\
    ... & ... & ... & ... & ... & ... & ... & ... & ...\\
    0 & 0 & 0 & 0 &  ... & -2 & 1 & 0 & 0\\
    0 & 0 & 0 & 0 &  ... & 1 & -2 & 1 & 0\\
    0 & 0 & 0 & 0 &  ... & 0 & 1 & -2 & 1\\
    0 & 0 & 0 & 0 &  ... & 0 & 0 & 1 & -2\\
    \end{pmatrix}
\end{equation}
where the diagonals are all -2 and sub-diagonals 1, and is called the difference matrix.

If you do the matrix multiplication, you will find that each row (other than the end row) in the matrix equation is exactly 
\begin{align}
    \frac{\phi_{i+1} - 2\phi_i + \phi_{i-1}}{h^2} &= \frac{\rho_i}{\epsilon_0},
\end{align}
i.e. the discretised form of the Maxwell's equation $eq.[\ref{maxwell3}]$! Hence, solving $eq.[\ref{matrix_equation}]$ gives a discretised solution to $eq.[\ref{maxwell3}]$! This is usually the way to efficiently solve a more complex differential equation. 

How do we solve $eq.[\ref{matrix_equation}]$? Learn linear algebra!


\end{document}
